\documentclass{beamer}
\usetheme{Warsaw}

%===============CORES===============
\setbeamercolor{normal text}{fg=white,bg=black!90}
\setbeamercolor{structure}{fg=white}
\setbeamercolor{alerted text}{fg=red!85!black}
\setbeamercolor{item projected}{use=item,fg=black,bg=item.fg!35}
\setbeamercolor*{palette primary}{use=structure,fg=structure.fg}
\setbeamercolor*{palette secondary}{use=structure,fg=structure.fg!95!black}
\setbeamercolor*{palette tertiary}{use=structure,fg=structure.fg!90!black}
\setbeamercolor*{palette quaternary}{use=structure,fg=structure.fg!95!black,bg=black!80}
\setbeamercolor*{framesubtitle}{fg=white}
\setbeamercolor*{block title}{parent=structure,bg=black!60}
\setbeamercolor*{block body}{fg=black,bg=black!10}
\setbeamercolor*{block title alerted}{parent=alerted text,bg=black!15}
\setbeamercolor*{block title example}{parent=example text,bg=black!15}
%===================================


\begin{document}

\begin{frame}
\frametitle{NR-7}
\framesubtitle{PROGRAMA DE CONTROLE MÉDICO DE SAÚDE OCUPACIONAL(PCMSO)}

%\begin{enumerate}
%\item Test
%\end{enumerate}

\begin{block}{O que é a NR-7?}
A NR-7 é uma norma regulamentadora que tem como objetivo obrigar os empregadores
a estabelecerem politicas de preservação de saúde para seus funcionarios. De acordo com esta norma, o PCMSO é responsável por avaliar os trabalhadores de forma individual ou coletiva, por meio de instrumentos clínico epidemiológicos.
\end{block}
\end{frame}

\begin{frame}
\frametitle{1}
\end{frame}

\begin{frame}
\frametitle{2}
\end{frame}

\begin{frame}
\frametitle{3 - O PCMSO}
\begin{block}{Qual o objetivo do PCMSO?}
O PCMSO tem caráter de prevenção e diagnóstico dos agravos à saúde do 
trabalhador. Nele consta também casos de doenças profissionais e danos 
irreversíveis à saúde do trabalhador.
O empregador tem como responsabilidade implementar de uma maneira efetiva o 
Programa de Controle Médico de Saúde Ocupacional (PCMSO), custeando sem ônus 
para os empregados os serviços do mesmo
\end{block}
\end{frame}

%=============
\begin{frame}
\frametitle{4 - A ASO}
\begin{block}{O que é a ASO}
ASO é a sigla para Atestado de Saúde Ocupacional. Este Atestado é emitido após 
a realização de qualquer exame médico ocupacional (admissão, demissional, 
periódico, de retorno ao trabalho, ou de mudança de função). O ASO é o atestado 
que define se o funcionário está apto ou inapto para a realização de suas 
funções dentro da empresa. Dentro da ASO está contido os dados do funcionário
\end{block}
\end{frame}
%=============

%=============
\begin{frame}
\frametitle{5 - Itens obrigatórios em um kit de primeiros socorros}
\begin{itemize}
    \item Algodão
    \item Máscara facial
    \item Luva de procedimento cirúrgico (descartável)
    \item Óculos de proteção
    \item Gaze
    \item Atadura de Crepom de 20 cm
    \item Pinça
    \item Curativo adesivo (tipo band-aid)
    \item Esparadrapos
    \item Bolsas térmicas, quentes e frias
    \item Talas
    \item Tesoura de ponta redonda
    \item Álcool
    \item Água oxigenada
    \item Frasco de Soro Fisiológico
    \item Sabão liquido bactericida
    \item Cotonete e pomidine pequeno
\end{itemize}
\end{frame}
%=============

\end{document}
